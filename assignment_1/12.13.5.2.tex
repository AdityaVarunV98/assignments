\documentclass[12pt,twocolumn,notitlepage]{article}
\usepackage[margin=0.5in]{geometry}
\usepackage{amsmath}
\usepackage{gensymb}
\usepackage{graphicx}
\usepackage{amsthm}
\usepackage{mathrsfs}
\usepackage{txfonts}
\usepackage{cite}
\usepackage{cases}
\usepackage{subfig}
\usepackage[breaklinks=true]{hyperref}
\usepackage{listings}
\usepackage[latin1]{inputenc}
\usepackage{color}
\usepackage{array}
\usepackage{longtable}
\usepackage{calc}
\usepackage{multirow}
\usepackage{hhline}
\usepackage{ifthen}
\usepackage{amssymb}

\newcommand*{\comb}[2]{{}^{#1}C_{#2}}

\title{Probability Assignment 1 (12.13.5.2)}
\author{Aditya Varun V (AI22BTECH11001)}
\date{}

\begin{document}
\maketitle
\subsection*{Question}
A pair of dice is thrown 4 times. If getting a doublet is considered a success, find
the probability of two successes.


\subsection*{Solution}

Let X denote the number of doublets/successes obtained after the 4 trials. Clearly, X has the binomial distribution with $n=4$ and
\begin{align}
    p &= \text{probability of getting a doublet with two dice} \nonumber \\
    &= \frac{1}{6} \nonumber
\end{align}

Now, since X has the binomial distribution, the probability mass function is given by
\begin{align}
    P(\text{exactly } r \text{ successes}) &= \comb{n}{r}\left(\frac{1}{6}\right)^{r}\left(\frac{5}{6}\right)^{n-r} \nonumber\\
    P_X(r) &= \comb{n}{r}\left(\frac{1}{6}\right)^{r}\left(\frac{5}{6}\right)^{n-r} \nonumber
\end{align}

Hence, the probability of two successes is
\begin{align}
    P_X(2) &= \comb{4}{2}\left(\frac{1}{6}\right)^{2}\left(\frac{5}{6}\right)^{2} \nonumber\\
    &= \frac{25}{216} \nonumber
\end{align}

\end{document}

